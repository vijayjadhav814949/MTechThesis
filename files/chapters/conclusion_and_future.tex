\chapter{Conclusion}
\pagestyle{fancy}
\fancyhead[LO]{\itshape\nouppercase{\rightmark}}

\hspace{0.2in} This project designed and implemented an Angular module to access RESTful Web Services. This module makes communicating with a REST-based Web Service simpler and in fewer lines of code. It takes care of the authentication, erros, URL construction.  It makes use of Angular's http module to make request over HTTP to a RESTful Web Service and uses HTTP's standard methods such as GET, PUT, POST etc. This module `\textbf{angular-rest-service}' is made available as a npm package on www.npmjs.org. This module can be installed in an Angular by executing `\textbf{npm install angular-rest-service}' from command line, which will download the latest version into the projects' nodule-modules directory. angular-rest-service suppports Basic (or Bearer) authentication, OAuth2 authentication methods. In the future work, additional authentication techniques can be added. 
